\documentclass{article}

\usepackage{amsmath}
\usepackage{amssymb}
\usepackage{amsfonts}
\usepackage{amsthm}
\usepackage[]{microtype}

\theoremstyle{plain}
\newtheorem{thm}{Theorem}[section]
\newtheorem{prop}[thm]{Proposition}
\theoremstyle{definition}
\newtheorem{defn}{Definition}[section]

\def\X{\mathcal{X}}
\def\R{\mathbb{R}}
\newcommand{\inner}[2]{\left\langle #1 \vert #2 \right\rangle}

\begin{document}
\title{Kernels, RKHS}
\author{}
\date{}
\maketitle

\section{Definitions}%
\label{sec:definition}

\begin{defn}[Positive Definite Kernel]
    Let $\X$ be a nonempty set, and $k:\X\times\X \to \R$ a symmetric function.
    $k$ is called a positive definite kernel if, given a finite number of
    points $x_1, \dots, x_m \in \X$, we have
    \[
         \forall c \in \R^m \quad
        \sum_{i=1}^{m} c_i \bar c_j k(x_i, x_j) \geq 0.
    \]
i.e., the Gram matrix $K = \big(k(x_i, x_j)\big)_{1 \leq i, j \leq m}$ is positive
semi-definite.
\end{defn}

\begin{defn}[Reproducing Kernel Hilbert Space]
    Let $\X$ be a nonempty set, and $ \mathcal{H}$ a Hilbert space of functions
    $f:\X\to\R$.
    $ \mathcal{H} $ is called a reproducing kernel Hilbert space endowed with
    the dot product $\inner{\cdot}{\cdot}$ if there exists a function
    with the following properties.
    \begin{enumerate}
        \item $k$ has the reproducing property
            \[
                \forall f\in \mathcal{H} \inner{f}{k(x,\cdot)} = f(x).
            \]
            In particular,
            \[
                \inner{k(x, \cdot)}{k(x', \cdot)} = k(x,x').
            \]
        \item $k$ spans $ \mathcal{H}$, i.e., $ \mathcal{H}=\overline{\operatorname{span}
            \{k(x,\cdot) \vert x\in \X\} } $
    \end{enumerate}
\end{defn}

\begin{defn}[Conditionally Positive Definite Matrix]
    A matrix $K \in \R^{m,m}$ is called a conditionally positive definite
    matrix if, for all $c \in \R^m$ such that $\sum_{i=1}^{m} c_i = 0$, we have
    \[
        \sum_{i=1}^{m} c_i \bar c_j K_{ij} \geq 0.
    \]
\end{defn}

\begin{defn}[Conditionally Positive Definite Kernel]
    A kernel $k:\X\times\X \to \R$ is called conditionally positive definite kernel if
    for any points $x_1, \dots, x_m \in \X$, the Gram matrix $K = \big(k(x_i,
    x_j)\big)_{1 \leq i, j \leq m}$ is conditionally positive definite
\end{defn}

\begin{prop}[Constructing PD Kernels from CPD Kernels]
    Let $x_0 \in \X$, and let $k$ be a symmetric kernel on $\X\times\X$, then
    \[
        \tilde k(x,x') := \frac{1}{2} \left(k(x, x') - k(x, x_0), - k(x_0, x') + k(x_0, x_0)\right)
    \]
    is positive definite if and only if $k$ is conditionally positive definite.
\end{prop}

\end{document}
