\documentclass{article}

\usepackage{amsmath}
\usepackage{amssymb}
\usepackage{amsfonts}
\usepackage{amsthm}
\usepackage{microtype}

\def\Z{\mathbb{Z}}
\def\R{\mathbb{R}}
\def\C{\mathbb{C}}

\begin{document}
\title{Introduction to Koopman operator theory of dynamical systems}
\author{}
\date{}
\maketitle

\section{Dynamical Systems}%
\label{sec:dynamical_systems}

A dynamical system consists of a set of \textit{states}  $S$ and a
\textit{rule}  for the evolution of the points of that set.\\
In the case of discrete time, this rule is represented by a map $T:S\to S$.
And the points $(x_t)_{t\in \Z}$ evolve over time such that
\begin{equation}
    x_{t+1} = T(x_t).
\end{equation}
In the case of continuous time, the evolution rule is represented by a
differential equation
\begin{equation}
    \dot x(t) = f(x).
    \label{diff_eq}
\end{equation}
In this case, we additionally define the \textit{flow map} $F^t:S\to S$ for
each $t > 0$ as the map taking the initial state to the one at time $t$.
\begin{equation}
    F^t(x_0) = x_t.
    \label{flow}
\end{equation}
This map satisfies the semi-group property, i.e.,
\begin{equation}
    \forall s, t > 0, F^t \circ F^s(x_0) = F^t(x_s) = x_{t+s} = F^{t+s}(x_0),
\end{equation}
where $\circ$ is the composition operator.

\section{Important Notions}%
\label{sec:important_notions}

\subsection{Fixed Point}%
\label{sub:fixed_point}

A fixed point is a point $x$ such that $f(x)=0$ (equivalently, $\forall t > 0,
F^t(x)=x$).
Fixed points correspond to equilibria of dynamical systems.
We usually take an interest in the stability of such points, i.e., whether
trajectories that start near the fixed point remain close to it over time or
not.

\subsection{Limit Cycle}%
\label{sub:limit_cycle}

Limit cycles are isolated closed curves in the state space which correspond to
time-periodic solutions of \eqref{diff_eq}.
They can be generalized to tori, which are associated with quasi-periodic
motion.

\subsection{Invariant Set}%
\label{sub:invariant_set}

An invariant set $B$ satisfies $F^t(B) \subset B$ for all $t > 0$, i.e., the
trajectories starting in $B$ remain in $B$.
Invariant sets are important because we can isolate them from the rest of the
space and limit ourselves to their dynamics.

\subsection{Attractor}%
\label{sub:attractor}

An attractor is an attracting set containing a dense orbit.
An attracting set is a subset to which many initial conditions converge.
A dense orbit is a trajectory that comes arbitrarily close to any point on the
attractor.
Attractors tend to determine the long term dynamics of dissipative systems.

\subsection{Bifurcation}%
\label{sub:bifurcation}

Bifurcation analysis is the study of changes in qualitative properties of the
trajectories due to changes in the vector field $f$ (or the map $T$),
such as the stability of fixed points or the existence of limit cycles.

Historically, dynamical systems have been studied by solving them analytically
or approximatively, then analyzing their dynamics.
But this has proven difficult when high-dimensional spaces are involved,
due to huge computational cost of solving the systems, as well as difficulties
in visualizing the solutions and their geometric properties.
Another issue is that the models that govern most dynamical systems in the real
world are sometimes uncertain or even completely unknown, a situation that the
classical tools aren't capable of dealing with.

This has pushed the field of dynamical analysis away from model-based
techniques to a more data-driven perspective, which is further reinforced by
the wide availability of data today.
The Koopman operator theory is a framework that manages to integrate data into
the mathematical analysis of dynamical systems.

\section{The Data-driven Viewpoint}%
\label{sec:the_data_driven_viewpoint}

In the context of dynamical systems, data is viewed as knowledge of some
variables related to the state of the system, i.e., functions of the state.
We call these functions \textit{observables} of the system.

The Koopman operator theory provides a new perspective on the problem, where
instead of considering the evolution of states, we consider the evolution
of observables.

Formally, we consider the vector space $\mathcal{F}$ of real-valued observables
from $S$ to $\R$.
Then the Koopman operator is a linear operator on this space, defined as
\begin{equation}
    U g(x) = g\circ T(x).
\end{equation}
In the case of continuous time systems, we define a semi-group of operators
$(U^t)_{t \geq 0}$, where each element is defined by
\begin{equation}
    U^t g (x) = g\circ F^t(x),
\end{equation}
where $F^t$ is the flow map defined in \eqref{flow}.

The advantage of the Koopman operator is that it models the evolution of a
nonlinear dynamical system by the evolution of linear system.
And the disadvantage is that it operates on an infinite dimensional space.

We hope to address this issue by discretizing the function space, and replacing
the Koopman operator by a finite dimensional approximation.

\section{Koopman linear expansion}%
\label{sec:koopman_linear_expansion}

Being a linear operator, we can gain deeper insight into the Koopman operator
by looking at its eigenvalues and eigenvectors.
Let $\phi_j:S\to \C$ be a complex-valued observable of the dynamical system
\eqref{diff_eq} and $\lambda_j$ a complex number.
We call the couple $(\phi_j, \lambda_j)$ an eigenfunction-eigenvalue pair of
the Koopman operator if
\begin{equation}
    \label{eig}
    U^t\phi_j = e^{\lambda_j t} \phi_j
\end{equation}
One interesting properties of the Koopman eigenfunctions is that if
$(\phi_i, \lambda_i)$ and $(\phi_j, \lambda_j)$ are eigenfunction-eigenvalue
pairs, then so is $(\phi_i \cdot \phi_j, \lambda_i + \lambda_j)$.

Let us assume that all the observables can be written as linear combinations of
such Koopman eigenfunctions, i.e.,
\begin{equation}
    g(x) = \sum_{k=0}^{\infty}g_k \phi_k(x).
\end{equation}
Then the evolution of the observable is given by
\begin{equation}
    U^t g(x) = \sum_{k=0}^{\infty}g_k e^{\lambda_k t} \phi_k(x).
\end{equation}
This expansion holds for a large class of nonlinear dynamical systems,
including ones with hyperbolic fixed points, limit cycles and tori.
\end{document}
