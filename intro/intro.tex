\documentclass{article}

\usepackage{amsmath}
\usepackage{amssymb}
\usepackage{amsfonts}
\usepackage{amsthm}

\def\Z{\mathbb{Z}}
\def\R{\mathbb{R}}

\begin{document}
\title{Introduction to Koopman operator theory of dynamical systems}
\author{}
\date{}
\maketitle


\section{Dynamical Systems}%
\label{sec:dynamical_systems}

A dynamical system consists of a set of \textit{states}  $S$ and a
\textit{rule}  for the evolution of the points of that set.\\
In the case of discrete time, this rule is represented by an operator $T:S\to S$.
And the points $(x_t)_{t\in \Z}$ evolve over time such that
\[
    x_{t+1} = T(x_t).
\]
In the case of continuous time, the evolution rule is represented by a
differential equation
\[
    \dot x(t) = f(x).
\]
In this case, we additionally define the \textit{flow map} $F^t:S\to S$ for
each $t > 0$ as the map taking the initial state to the one at time $t$.
\[
    F^t(x_0) = x_t.
\]
This map satisfies the semi-group property, i.e.,
\begin{align*}
    \forall s, t > 0, F^t \circ F^s(x_0) = F^t(x_s) = x_{t+s} = F^{t+s}(x_0),
\end{align*}
where $\circ$ is the composition operator.

\section{Important notions}%
\label{sec:important_notions}

\subsection{Fixed point}%
\label{sub:fixed_point}

abc
\end{document}
